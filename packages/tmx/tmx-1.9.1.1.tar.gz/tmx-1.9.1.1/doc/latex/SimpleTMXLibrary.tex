% Generated by Sphinx.
\def\sphinxdocclass{report}
\documentclass[letterpaper,10pt,english]{sphinxmanual}
\usepackage[utf8]{inputenc}
\DeclareUnicodeCharacter{00A0}{\nobreakspace}
\usepackage{cmap}
\usepackage[T1]{fontenc}

\usepackage{babel}
\usepackage{times}
\usepackage[Bjarne]{fncychap}
\usepackage{longtable}
\usepackage{sphinx}
\usepackage{multirow}
\usepackage{eqparbox}


\addto\captionsenglish{\renewcommand{\figurename}{Fig. }}
\addto\captionsenglish{\renewcommand{\tablename}{Table }}
\SetupFloatingEnvironment{literal-block}{name=Listing }



\title{Simple TMX Library Documentation}
\date{December 15, 2016}
\release{1.9.1}
\author{Julie Marchant}
\newcommand{\sphinxlogo}{}
\renewcommand{\releasename}{Release}

\makeindex

\makeatletter
\def\PYG@reset{\let\PYG@it=\relax \let\PYG@bf=\relax%
    \let\PYG@ul=\relax \let\PYG@tc=\relax%
    \let\PYG@bc=\relax \let\PYG@ff=\relax}
\def\PYG@tok#1{\csname PYG@tok@#1\endcsname}
\def\PYG@toks#1+{\ifx\relax#1\empty\else%
    \PYG@tok{#1}\expandafter\PYG@toks\fi}
\def\PYG@do#1{\PYG@bc{\PYG@tc{\PYG@ul{%
    \PYG@it{\PYG@bf{\PYG@ff{#1}}}}}}}
\def\PYG#1#2{\PYG@reset\PYG@toks#1+\relax+\PYG@do{#2}}

\expandafter\def\csname PYG@tok@bp\endcsname{\def\PYG@tc##1{\textcolor[rgb]{0.00,0.44,0.13}{##1}}}
\expandafter\def\csname PYG@tok@gh\endcsname{\let\PYG@bf=\textbf\def\PYG@tc##1{\textcolor[rgb]{0.00,0.00,0.50}{##1}}}
\expandafter\def\csname PYG@tok@se\endcsname{\let\PYG@bf=\textbf\def\PYG@tc##1{\textcolor[rgb]{0.25,0.44,0.63}{##1}}}
\expandafter\def\csname PYG@tok@na\endcsname{\def\PYG@tc##1{\textcolor[rgb]{0.25,0.44,0.63}{##1}}}
\expandafter\def\csname PYG@tok@sc\endcsname{\def\PYG@tc##1{\textcolor[rgb]{0.25,0.44,0.63}{##1}}}
\expandafter\def\csname PYG@tok@sr\endcsname{\def\PYG@tc##1{\textcolor[rgb]{0.14,0.33,0.53}{##1}}}
\expandafter\def\csname PYG@tok@sx\endcsname{\def\PYG@tc##1{\textcolor[rgb]{0.78,0.36,0.04}{##1}}}
\expandafter\def\csname PYG@tok@gt\endcsname{\def\PYG@tc##1{\textcolor[rgb]{0.00,0.27,0.87}{##1}}}
\expandafter\def\csname PYG@tok@vg\endcsname{\def\PYG@tc##1{\textcolor[rgb]{0.73,0.38,0.84}{##1}}}
\expandafter\def\csname PYG@tok@cp\endcsname{\def\PYG@tc##1{\textcolor[rgb]{0.00,0.44,0.13}{##1}}}
\expandafter\def\csname PYG@tok@w\endcsname{\def\PYG@tc##1{\textcolor[rgb]{0.73,0.73,0.73}{##1}}}
\expandafter\def\csname PYG@tok@ow\endcsname{\let\PYG@bf=\textbf\def\PYG@tc##1{\textcolor[rgb]{0.00,0.44,0.13}{##1}}}
\expandafter\def\csname PYG@tok@s\endcsname{\def\PYG@tc##1{\textcolor[rgb]{0.25,0.44,0.63}{##1}}}
\expandafter\def\csname PYG@tok@s1\endcsname{\def\PYG@tc##1{\textcolor[rgb]{0.25,0.44,0.63}{##1}}}
\expandafter\def\csname PYG@tok@cm\endcsname{\let\PYG@it=\textit\def\PYG@tc##1{\textcolor[rgb]{0.25,0.50,0.56}{##1}}}
\expandafter\def\csname PYG@tok@si\endcsname{\let\PYG@it=\textit\def\PYG@tc##1{\textcolor[rgb]{0.44,0.63,0.82}{##1}}}
\expandafter\def\csname PYG@tok@gu\endcsname{\let\PYG@bf=\textbf\def\PYG@tc##1{\textcolor[rgb]{0.50,0.00,0.50}{##1}}}
\expandafter\def\csname PYG@tok@vc\endcsname{\def\PYG@tc##1{\textcolor[rgb]{0.73,0.38,0.84}{##1}}}
\expandafter\def\csname PYG@tok@sd\endcsname{\let\PYG@it=\textit\def\PYG@tc##1{\textcolor[rgb]{0.25,0.44,0.63}{##1}}}
\expandafter\def\csname PYG@tok@gi\endcsname{\def\PYG@tc##1{\textcolor[rgb]{0.00,0.63,0.00}{##1}}}
\expandafter\def\csname PYG@tok@ss\endcsname{\def\PYG@tc##1{\textcolor[rgb]{0.32,0.47,0.09}{##1}}}
\expandafter\def\csname PYG@tok@sh\endcsname{\def\PYG@tc##1{\textcolor[rgb]{0.25,0.44,0.63}{##1}}}
\expandafter\def\csname PYG@tok@no\endcsname{\def\PYG@tc##1{\textcolor[rgb]{0.38,0.68,0.84}{##1}}}
\expandafter\def\csname PYG@tok@nv\endcsname{\def\PYG@tc##1{\textcolor[rgb]{0.73,0.38,0.84}{##1}}}
\expandafter\def\csname PYG@tok@gr\endcsname{\def\PYG@tc##1{\textcolor[rgb]{1.00,0.00,0.00}{##1}}}
\expandafter\def\csname PYG@tok@mo\endcsname{\def\PYG@tc##1{\textcolor[rgb]{0.13,0.50,0.31}{##1}}}
\expandafter\def\csname PYG@tok@m\endcsname{\def\PYG@tc##1{\textcolor[rgb]{0.13,0.50,0.31}{##1}}}
\expandafter\def\csname PYG@tok@vi\endcsname{\def\PYG@tc##1{\textcolor[rgb]{0.73,0.38,0.84}{##1}}}
\expandafter\def\csname PYG@tok@err\endcsname{\def\PYG@bc##1{\setlength{\fboxsep}{0pt}\fcolorbox[rgb]{1.00,0.00,0.00}{1,1,1}{\strut ##1}}}
\expandafter\def\csname PYG@tok@kt\endcsname{\def\PYG@tc##1{\textcolor[rgb]{0.56,0.13,0.00}{##1}}}
\expandafter\def\csname PYG@tok@kc\endcsname{\let\PYG@bf=\textbf\def\PYG@tc##1{\textcolor[rgb]{0.00,0.44,0.13}{##1}}}
\expandafter\def\csname PYG@tok@ni\endcsname{\let\PYG@bf=\textbf\def\PYG@tc##1{\textcolor[rgb]{0.84,0.33,0.22}{##1}}}
\expandafter\def\csname PYG@tok@mb\endcsname{\def\PYG@tc##1{\textcolor[rgb]{0.13,0.50,0.31}{##1}}}
\expandafter\def\csname PYG@tok@nl\endcsname{\let\PYG@bf=\textbf\def\PYG@tc##1{\textcolor[rgb]{0.00,0.13,0.44}{##1}}}
\expandafter\def\csname PYG@tok@kd\endcsname{\let\PYG@bf=\textbf\def\PYG@tc##1{\textcolor[rgb]{0.00,0.44,0.13}{##1}}}
\expandafter\def\csname PYG@tok@nd\endcsname{\let\PYG@bf=\textbf\def\PYG@tc##1{\textcolor[rgb]{0.33,0.33,0.33}{##1}}}
\expandafter\def\csname PYG@tok@k\endcsname{\let\PYG@bf=\textbf\def\PYG@tc##1{\textcolor[rgb]{0.00,0.44,0.13}{##1}}}
\expandafter\def\csname PYG@tok@mh\endcsname{\def\PYG@tc##1{\textcolor[rgb]{0.13,0.50,0.31}{##1}}}
\expandafter\def\csname PYG@tok@cpf\endcsname{\let\PYG@it=\textit\def\PYG@tc##1{\textcolor[rgb]{0.25,0.50,0.56}{##1}}}
\expandafter\def\csname PYG@tok@ne\endcsname{\def\PYG@tc##1{\textcolor[rgb]{0.00,0.44,0.13}{##1}}}
\expandafter\def\csname PYG@tok@s2\endcsname{\def\PYG@tc##1{\textcolor[rgb]{0.25,0.44,0.63}{##1}}}
\expandafter\def\csname PYG@tok@cs\endcsname{\def\PYG@tc##1{\textcolor[rgb]{0.25,0.50,0.56}{##1}}\def\PYG@bc##1{\setlength{\fboxsep}{0pt}\colorbox[rgb]{1.00,0.94,0.94}{\strut ##1}}}
\expandafter\def\csname PYG@tok@ch\endcsname{\let\PYG@it=\textit\def\PYG@tc##1{\textcolor[rgb]{0.25,0.50,0.56}{##1}}}
\expandafter\def\csname PYG@tok@gs\endcsname{\let\PYG@bf=\textbf}
\expandafter\def\csname PYG@tok@nc\endcsname{\let\PYG@bf=\textbf\def\PYG@tc##1{\textcolor[rgb]{0.05,0.52,0.71}{##1}}}
\expandafter\def\csname PYG@tok@kr\endcsname{\let\PYG@bf=\textbf\def\PYG@tc##1{\textcolor[rgb]{0.00,0.44,0.13}{##1}}}
\expandafter\def\csname PYG@tok@sb\endcsname{\def\PYG@tc##1{\textcolor[rgb]{0.25,0.44,0.63}{##1}}}
\expandafter\def\csname PYG@tok@mi\endcsname{\def\PYG@tc##1{\textcolor[rgb]{0.13,0.50,0.31}{##1}}}
\expandafter\def\csname PYG@tok@nb\endcsname{\def\PYG@tc##1{\textcolor[rgb]{0.00,0.44,0.13}{##1}}}
\expandafter\def\csname PYG@tok@nn\endcsname{\let\PYG@bf=\textbf\def\PYG@tc##1{\textcolor[rgb]{0.05,0.52,0.71}{##1}}}
\expandafter\def\csname PYG@tok@c\endcsname{\let\PYG@it=\textit\def\PYG@tc##1{\textcolor[rgb]{0.25,0.50,0.56}{##1}}}
\expandafter\def\csname PYG@tok@c1\endcsname{\let\PYG@it=\textit\def\PYG@tc##1{\textcolor[rgb]{0.25,0.50,0.56}{##1}}}
\expandafter\def\csname PYG@tok@kn\endcsname{\let\PYG@bf=\textbf\def\PYG@tc##1{\textcolor[rgb]{0.00,0.44,0.13}{##1}}}
\expandafter\def\csname PYG@tok@gd\endcsname{\def\PYG@tc##1{\textcolor[rgb]{0.63,0.00,0.00}{##1}}}
\expandafter\def\csname PYG@tok@nt\endcsname{\let\PYG@bf=\textbf\def\PYG@tc##1{\textcolor[rgb]{0.02,0.16,0.45}{##1}}}
\expandafter\def\csname PYG@tok@o\endcsname{\def\PYG@tc##1{\textcolor[rgb]{0.40,0.40,0.40}{##1}}}
\expandafter\def\csname PYG@tok@kp\endcsname{\def\PYG@tc##1{\textcolor[rgb]{0.00,0.44,0.13}{##1}}}
\expandafter\def\csname PYG@tok@ge\endcsname{\let\PYG@it=\textit}
\expandafter\def\csname PYG@tok@il\endcsname{\def\PYG@tc##1{\textcolor[rgb]{0.13,0.50,0.31}{##1}}}
\expandafter\def\csname PYG@tok@gp\endcsname{\let\PYG@bf=\textbf\def\PYG@tc##1{\textcolor[rgb]{0.78,0.36,0.04}{##1}}}
\expandafter\def\csname PYG@tok@go\endcsname{\def\PYG@tc##1{\textcolor[rgb]{0.20,0.20,0.20}{##1}}}
\expandafter\def\csname PYG@tok@mf\endcsname{\def\PYG@tc##1{\textcolor[rgb]{0.13,0.50,0.31}{##1}}}
\expandafter\def\csname PYG@tok@nf\endcsname{\def\PYG@tc##1{\textcolor[rgb]{0.02,0.16,0.49}{##1}}}

\def\PYGZbs{\char`\\}
\def\PYGZus{\char`\_}
\def\PYGZob{\char`\{}
\def\PYGZcb{\char`\}}
\def\PYGZca{\char`\^}
\def\PYGZam{\char`\&}
\def\PYGZlt{\char`\<}
\def\PYGZgt{\char`\>}
\def\PYGZsh{\char`\#}
\def\PYGZpc{\char`\%}
\def\PYGZdl{\char`\$}
\def\PYGZhy{\char`\-}
\def\PYGZsq{\char`\'}
\def\PYGZdq{\char`\"}
\def\PYGZti{\char`\~}
% for compatibility with earlier versions
\def\PYGZat{@}
\def\PYGZlb{[}
\def\PYGZrb{]}
\makeatother

\renewcommand\PYGZsq{\textquotesingle}

\begin{document}

\maketitle
\tableofcontents
\phantomsection\label{index::doc}

\phantomsection\label{index:module-tmx}\index{tmx (module)}
This library reads and writes the Tiled TMX format in a simple way.
This is useful for map editors or generic level editors, and it's also
useful for using a map editor or generic level editor like Tiled to edit
your game's levels.

To load a TMX file, use {\hyperref[index:tmx.TileMap.load]{\emph{\code{tmx.TileMap.load()}}}}.  You can then read the
attributes of the returned {\hyperref[index:tmx.TileMap]{\emph{\code{tmx.TileMap}}}} object, modify the
attributes to your liking, and save your changes with
{\hyperref[index:tmx.TileMap.save]{\emph{\code{tmx.TileMap.save()}}}}.  That's it!  Simple, isn't it?

At the request of the developer of Tiled, this documentation does not
explain in detail what each attribute means. For that, please see the
TMX format specification, found here:

\href{http://doc.mapeditor.org/reference/tmx-map-format/}{http://doc.mapeditor.org/reference/tmx-map-format/}


\chapter{tmx.TileMap}
\label{index:tmx-tilemap}\label{index:simple-tmx-library-documentation}\index{TileMap (class in tmx)}

\begin{fulllineitems}
\phantomsection\label{index:tmx.TileMap}\pysigline{\strong{class }\code{tmx.}\bfcode{TileMap}}
This class loads, stores, and saves TMX files.
\index{version (tmx.TileMap attribute)}

\begin{fulllineitems}
\phantomsection\label{index:tmx.TileMap.version}\pysigline{\bfcode{version}}
The TMX format version.

\end{fulllineitems}

\index{orientation (tmx.TileMap attribute)}

\begin{fulllineitems}
\phantomsection\label{index:tmx.TileMap.orientation}\pysigline{\bfcode{orientation}}
Map orientation.  Can be ``orthogonal'', ``isometric'', ``staggered'',
or ``hexagonal''.

\end{fulllineitems}

\index{renderorder (tmx.TileMap attribute)}

\begin{fulllineitems}
\phantomsection\label{index:tmx.TileMap.renderorder}\pysigline{\bfcode{renderorder}}
The order in which tiles are rendered.  Can be \code{"right-down"},
\code{"right-up"}, \code{"left-down"}, or \code{"left-up"}.  Default is
\code{"right-down"}.

\end{fulllineitems}

\index{width (tmx.TileMap attribute)}

\begin{fulllineitems}
\phantomsection\label{index:tmx.TileMap.width}\pysigline{\bfcode{width}}
The width of the map in tiles.

\end{fulllineitems}

\index{height (tmx.TileMap attribute)}

\begin{fulllineitems}
\phantomsection\label{index:tmx.TileMap.height}\pysigline{\bfcode{height}}
The height of the map in tiles.

\end{fulllineitems}

\index{tilewidth (tmx.TileMap attribute)}

\begin{fulllineitems}
\phantomsection\label{index:tmx.TileMap.tilewidth}\pysigline{\bfcode{tilewidth}}
The width of a tile.

\end{fulllineitems}

\index{tileheight (tmx.TileMap attribute)}

\begin{fulllineitems}
\phantomsection\label{index:tmx.TileMap.tileheight}\pysigline{\bfcode{tileheight}}
The height of a tile.

\end{fulllineitems}

\index{staggeraxis (tmx.TileMap attribute)}

\begin{fulllineitems}
\phantomsection\label{index:tmx.TileMap.staggeraxis}\pysigline{\bfcode{staggeraxis}}
Determines which axis is staggered.  Can be ``x'' or ``y''.  Set to
\code{None} to not set it.  Only meaningful for staggered and
hexagonal maps.

\end{fulllineitems}

\index{staggerindex (tmx.TileMap attribute)}

\begin{fulllineitems}
\phantomsection\label{index:tmx.TileMap.staggerindex}\pysigline{\bfcode{staggerindex}}
Determines what indexes along the staggered axis are shifted.
Can be ``even'' or ``odd''.  Set to \code{None} to not set it.
Only meaningful for staggered and hexagonal maps.

\end{fulllineitems}

\index{hexsidelength (tmx.TileMap attribute)}

\begin{fulllineitems}
\phantomsection\label{index:tmx.TileMap.hexsidelength}\pysigline{\bfcode{hexsidelength}}
Side length of the hexagon in hexagonal tiles.  Set to
\code{None} to not set it.  Only meaningful for hexagonal maps.

\end{fulllineitems}

\index{backgroundcolor (tmx.TileMap attribute)}

\begin{fulllineitems}
\phantomsection\label{index:tmx.TileMap.backgroundcolor}\pysigline{\bfcode{backgroundcolor}}
A {\hyperref[index:tmx.Color]{\emph{\code{Color}}}} object indicating the background color of the
map, or \code{None} if no background color is defined.

\end{fulllineitems}

\index{nextobjectid (tmx.TileMap attribute)}

\begin{fulllineitems}
\phantomsection\label{index:tmx.TileMap.nextobjectid}\pysigline{\bfcode{nextobjectid}}
The next available ID for new objects.  Set to \code{None} to
not set it.

\end{fulllineitems}

\index{properties (tmx.TileMap attribute)}

\begin{fulllineitems}
\phantomsection\label{index:tmx.TileMap.properties}\pysigline{\bfcode{properties}}
A list of {\hyperref[index:tmx.Property]{\emph{\code{Property}}}} objects indicating the map's
properties.

\end{fulllineitems}

\index{tilesets (tmx.TileMap attribute)}

\begin{fulllineitems}
\phantomsection\label{index:tmx.TileMap.tilesets}\pysigline{\bfcode{tilesets}}
A list of {\hyperref[index:tmx.Tileset]{\emph{\code{Tileset}}}} objects indicating the map's tilesets.

\end{fulllineitems}

\index{layers (tmx.TileMap attribute)}

\begin{fulllineitems}
\phantomsection\label{index:tmx.TileMap.layers}\pysigline{\bfcode{layers}}
A list of {\hyperref[index:tmx.Layer]{\emph{\code{Layer}}}}, {\hyperref[index:tmx.ObjectGroup]{\emph{\code{ObjectGroup}}}}, and
{\hyperref[index:tmx.ImageLayer]{\emph{\code{ImageLayer}}}} objects indicating the map's tile layers,
object groups, and image layers, respectively.  Those that appear
in this list first are rendered first (i.e. furthest in the
back).

\end{fulllineitems}


\end{fulllineitems}

\index{load() (tmx.TileMap class method)}

\begin{fulllineitems}
\phantomsection\label{index:tmx.TileMap.load}\pysiglinewithargsret{\strong{classmethod }\code{TileMap.}\bfcode{load}}{\emph{fname}}{}
Load the TMX file with the indicated name and return a
{\hyperref[index:tmx.TileMap]{\emph{\code{TileMap}}}} object representing it.

\end{fulllineitems}

\index{save() (tmx.TileMap method)}

\begin{fulllineitems}
\phantomsection\label{index:tmx.TileMap.save}\pysiglinewithargsret{\code{TileMap.}\bfcode{save}}{\emph{fname}, \emph{data\_encoding='base64'}, \emph{data\_compression=True}}{}
Save the object to the file with the indicated name.

Arguments:
\begin{itemize}
\item {} 
\code{data\_encoding} -- The encoding to use for layers.  Can be
\code{"base64"} or \code{"csv"}.  Set to \code{None} for the
default encoding (currently \code{"base64"}).

\item {} 
\code{data\_compression} -- Whether or not compression should be
used on layers if possible (currently only possible for
base64-encoded data).

\end{itemize}

\end{fulllineitems}



\chapter{Other Classes}
\label{index:other-classes}\index{Color (class in tmx)}

\begin{fulllineitems}
\phantomsection\label{index:tmx.Color}\pysiglinewithargsret{\strong{class }\code{tmx.}\bfcode{Color}}{\emph{hex\_string='\#000000'}}{}~\index{red (tmx.Color attribute)}

\begin{fulllineitems}
\phantomsection\label{index:tmx.Color.red}\pysigline{\bfcode{red}}
The red component of the color as an integer, where \code{0}
indicates no red intensity and \code{255} indicates full red
intensity.

\end{fulllineitems}

\index{green (tmx.Color attribute)}

\begin{fulllineitems}
\phantomsection\label{index:tmx.Color.green}\pysigline{\bfcode{green}}
The green component of the color as an integer, where \code{0}
indicates no green intensity and \code{255} indicates full green
intensity.

\end{fulllineitems}

\index{blue (tmx.Color attribute)}

\begin{fulllineitems}
\phantomsection\label{index:tmx.Color.blue}\pysigline{\bfcode{blue}}
The blue component of the color as an integer, where \code{0}
indicates no blue intensity and \code{255} indicates full blue
intensity.

\end{fulllineitems}

\index{alpha (tmx.Color attribute)}

\begin{fulllineitems}
\phantomsection\label{index:tmx.Color.alpha}\pysigline{\bfcode{alpha}}
The alpha transparency of the color as an integer, where \code{0}
indicates full transparency and \code{255} indicates full opacity.

\end{fulllineitems}

\index{hex\_string (tmx.Color attribute)}

\begin{fulllineitems}
\phantomsection\label{index:tmx.Color.hex_string}\pysigline{\bfcode{hex\_string}}
The hex string representation of the color used by the TMX file.
The format of the string is either \code{"\#RRGGBB"} or
\code{"\#AARRGGBB"}.  The hash at the beginning is optional.

\end{fulllineitems}


\end{fulllineitems}

\index{Image (class in tmx)}

\begin{fulllineitems}
\phantomsection\label{index:tmx.Image}\pysiglinewithargsret{\strong{class }\code{tmx.}\bfcode{Image}}{\emph{format\_=None}, \emph{source=None}, \emph{trans=None}, \emph{width=None}, \emph{height=None}, \emph{data=None}}{}~\index{format (tmx.Image attribute)}

\begin{fulllineitems}
\phantomsection\label{index:tmx.Image.format}\pysigline{\bfcode{format}}
Indicates the format of image data if embedded.  Should be an
extension like \code{"png"}, \code{"gif"}, \code{"jpg"}, or \code{"bmp"}.
Set to \code{None} to not specify the format.

\end{fulllineitems}

\index{source (tmx.Image attribute)}

\begin{fulllineitems}
\phantomsection\label{index:tmx.Image.source}\pysigline{\bfcode{source}}
The location of the image file referenced.  If set to
\code{None}, the image data is embedded.

\end{fulllineitems}

\index{trans (tmx.Image attribute)}

\begin{fulllineitems}
\phantomsection\label{index:tmx.Image.trans}\pysigline{\bfcode{trans}}
A {\hyperref[index:tmx.Color]{\emph{\code{Color}}}} object indicating the transparent color of the
image, or \code{None} if no color is treated as transparent.

\end{fulllineitems}

\index{width (tmx.Image attribute)}

\begin{fulllineitems}
\phantomsection\label{index:tmx.Image.width}\pysigline{\bfcode{width}}
The width of the image in pixels; used for tile index correction
when the image changes.  If set to \code{None}, the image width
is not explicitly specified.

\end{fulllineitems}

\index{height (tmx.Image attribute)}

\begin{fulllineitems}
\phantomsection\label{index:tmx.Image.height}\pysigline{\bfcode{height}}
The height of the image in pixels; used for tile index correction
when the image changes.  If set to \code{None}, the image
height is not explicitly specified.

\end{fulllineitems}

\index{data (tmx.Image attribute)}

\begin{fulllineitems}
\phantomsection\label{index:tmx.Image.data}\pysigline{\bfcode{data}}
The image data if embedded, or \code{None} if an external image
is referenced.

\end{fulllineitems}


\end{fulllineitems}

\index{ImageLayer (class in tmx)}

\begin{fulllineitems}
\phantomsection\label{index:tmx.ImageLayer}\pysiglinewithargsret{\strong{class }\code{tmx.}\bfcode{ImageLayer}}{\emph{name}, \emph{offsetx}, \emph{offsety}, \emph{opacity=1}, \emph{visible=True}, \emph{properties=None}, \emph{image=None}}{}~\index{name (tmx.ImageLayer attribute)}

\begin{fulllineitems}
\phantomsection\label{index:tmx.ImageLayer.name}\pysigline{\bfcode{name}}
The name of the image layer.

\end{fulllineitems}

\index{offsetx (tmx.ImageLayer attribute)}

\begin{fulllineitems}
\phantomsection\label{index:tmx.ImageLayer.offsetx}\pysigline{\bfcode{offsetx}}
The x position of the image layer in pixels.

\end{fulllineitems}

\index{offsety (tmx.ImageLayer attribute)}

\begin{fulllineitems}
\phantomsection\label{index:tmx.ImageLayer.offsety}\pysigline{\bfcode{offsety}}
The y position of the image layer in pixels.

\end{fulllineitems}

\index{opacity (tmx.ImageLayer attribute)}

\begin{fulllineitems}
\phantomsection\label{index:tmx.ImageLayer.opacity}\pysigline{\bfcode{opacity}}
The opacity of the image layer as a value from 0 to 1.

\end{fulllineitems}

\index{visible (tmx.ImageLayer attribute)}

\begin{fulllineitems}
\phantomsection\label{index:tmx.ImageLayer.visible}\pysigline{\bfcode{visible}}
Whether or not the image layer is visible.

\end{fulllineitems}

\index{properties (tmx.ImageLayer attribute)}

\begin{fulllineitems}
\phantomsection\label{index:tmx.ImageLayer.properties}\pysigline{\bfcode{properties}}
A list of {\hyperref[index:tmx.Property]{\emph{\code{Property}}}} objects indicating the properties of
the image layer.

\end{fulllineitems}

\index{image (tmx.ImageLayer attribute)}

\begin{fulllineitems}
\phantomsection\label{index:tmx.ImageLayer.image}\pysigline{\bfcode{image}}
An {\hyperref[index:tmx.Image]{\emph{\code{Image}}}} object indicating the image of the image layer.

\end{fulllineitems}


\end{fulllineitems}

\index{Layer (class in tmx)}

\begin{fulllineitems}
\phantomsection\label{index:tmx.Layer}\pysiglinewithargsret{\strong{class }\code{tmx.}\bfcode{Layer}}{\emph{name}, \emph{opacity=1}, \emph{visible=True}, \emph{offsetx=0}, \emph{offsety=0}, \emph{properties=None}, \emph{tiles=None}}{}~\index{name (tmx.Layer attribute)}

\begin{fulllineitems}
\phantomsection\label{index:tmx.Layer.name}\pysigline{\bfcode{name}}
The name of the layer.

\end{fulllineitems}

\index{opacity (tmx.Layer attribute)}

\begin{fulllineitems}
\phantomsection\label{index:tmx.Layer.opacity}\pysigline{\bfcode{opacity}}
The opacity of the layer as a value from 0 to 1.

\end{fulllineitems}

\index{visible (tmx.Layer attribute)}

\begin{fulllineitems}
\phantomsection\label{index:tmx.Layer.visible}\pysigline{\bfcode{visible}}
Whether or not the layer is visible.

\end{fulllineitems}

\index{offsetx (tmx.Layer attribute)}

\begin{fulllineitems}
\phantomsection\label{index:tmx.Layer.offsetx}\pysigline{\bfcode{offsetx}}
Rendering offset for this layer in pixels.

\end{fulllineitems}

\index{offsety (tmx.Layer attribute)}

\begin{fulllineitems}
\phantomsection\label{index:tmx.Layer.offsety}\pysigline{\bfcode{offsety}}
Rendering offset for this layer in pixels.

\end{fulllineitems}

\index{properties (tmx.Layer attribute)}

\begin{fulllineitems}
\phantomsection\label{index:tmx.Layer.properties}\pysigline{\bfcode{properties}}
A list of {\hyperref[index:tmx.Property]{\emph{\code{Property}}}} objects indicating the properties of
the layer.

\end{fulllineitems}

\index{tiles (tmx.Layer attribute)}

\begin{fulllineitems}
\phantomsection\label{index:tmx.Layer.tiles}\pysigline{\bfcode{tiles}}
A list of {\hyperref[index:tmx.LayerTile]{\emph{\code{LayerTile}}}} objects indicating the tiles of the
layer.

The coordinates of each tile is determined by the tile's index
within this list.  Exactly how the tiles are positioned is
determined by the map orientation.

\end{fulllineitems}


\end{fulllineitems}

\index{LayerTile (class in tmx)}

\begin{fulllineitems}
\phantomsection\label{index:tmx.LayerTile}\pysiglinewithargsret{\strong{class }\code{tmx.}\bfcode{LayerTile}}{\emph{gid}, \emph{hflip=False}, \emph{vflip=False}, \emph{dflip=False}}{}~\index{gid (tmx.LayerTile attribute)}

\begin{fulllineitems}
\phantomsection\label{index:tmx.LayerTile.gid}\pysigline{\bfcode{gid}}
The global ID of the tile.  A value of \code{0} indicates no tile at
this position.

\end{fulllineitems}

\index{hflip (tmx.LayerTile attribute)}

\begin{fulllineitems}
\phantomsection\label{index:tmx.LayerTile.hflip}\pysigline{\bfcode{hflip}}
Whether or not the tile is flipped horizontally.

\end{fulllineitems}

\index{vflip (tmx.LayerTile attribute)}

\begin{fulllineitems}
\phantomsection\label{index:tmx.LayerTile.vflip}\pysigline{\bfcode{vflip}}
Whether or not the tile is flipped vertically.

\end{fulllineitems}

\index{dflip (tmx.LayerTile attribute)}

\begin{fulllineitems}
\phantomsection\label{index:tmx.LayerTile.dflip}\pysigline{\bfcode{dflip}}
Whether or not the tile is flipped diagonally (X and Y axis
swapped).

\end{fulllineitems}


\end{fulllineitems}

\index{Object (class in tmx)}

\begin{fulllineitems}
\phantomsection\label{index:tmx.Object}\pysiglinewithargsret{\strong{class }\code{tmx.}\bfcode{Object}}{\emph{name}, \emph{type\_}, \emph{x}, \emph{y}, \emph{width=0}, \emph{height=0}, \emph{rotation=0}, \emph{gid=None}, \emph{visible=True}, \emph{properties=None}, \emph{ellipse=False}, \emph{polygon=None}, \emph{polyline=None}, \emph{id\_=None}}{}~\index{id (tmx.Object attribute)}

\begin{fulllineitems}
\phantomsection\label{index:tmx.Object.id}\pysigline{\bfcode{id}}
Unique ID of the object as a string if set, or \code{None}
otherwise.

\end{fulllineitems}

\index{name (tmx.Object attribute)}

\begin{fulllineitems}
\phantomsection\label{index:tmx.Object.name}\pysigline{\bfcode{name}}
The name of the object.  An arbitrary string.

\end{fulllineitems}

\index{type (tmx.Object attribute)}

\begin{fulllineitems}
\phantomsection\label{index:tmx.Object.type}\pysigline{\bfcode{type}}
The type of the object.  An arbitrary string.

\end{fulllineitems}

\index{x (tmx.Object attribute)}

\begin{fulllineitems}
\phantomsection\label{index:tmx.Object.x}\pysigline{\bfcode{x}}
The x coordinate of the object in pixels.  This is the
left edge of the object in orthogonal orientation, and the center
of the object otherwise.

\end{fulllineitems}

\index{y (tmx.Object attribute)}

\begin{fulllineitems}
\phantomsection\label{index:tmx.Object.y}\pysigline{\bfcode{y}}
The y coordinate of the object in pixels.  This is the bottom
edge of the object.

\end{fulllineitems}

\index{width (tmx.Object attribute)}

\begin{fulllineitems}
\phantomsection\label{index:tmx.Object.width}\pysigline{\bfcode{width}}
The width of the object in pixels.

\end{fulllineitems}

\index{height (tmx.Object attribute)}

\begin{fulllineitems}
\phantomsection\label{index:tmx.Object.height}\pysigline{\bfcode{height}}
The height of the object in pixels.

\end{fulllineitems}

\index{rotation (tmx.Object attribute)}

\begin{fulllineitems}
\phantomsection\label{index:tmx.Object.rotation}\pysigline{\bfcode{rotation}}
The rotation of the object in degrees clockwise.

\end{fulllineitems}

\index{gid (tmx.Object attribute)}

\begin{fulllineitems}
\phantomsection\label{index:tmx.Object.gid}\pysigline{\bfcode{gid}}
The tile to use as the object's image.  Set to \code{None} for
no reference to a tile.

\end{fulllineitems}

\index{visible (tmx.Object attribute)}

\begin{fulllineitems}
\phantomsection\label{index:tmx.Object.visible}\pysigline{\bfcode{visible}}
Whether or not the object is visible.

\end{fulllineitems}

\index{properties (tmx.Object attribute)}

\begin{fulllineitems}
\phantomsection\label{index:tmx.Object.properties}\pysigline{\bfcode{properties}}
A list of {\hyperref[index:tmx.Property]{\emph{\code{Property}}}} objects indicating the object's
properties.

\end{fulllineitems}

\index{ellipse (tmx.Object attribute)}

\begin{fulllineitems}
\phantomsection\label{index:tmx.Object.ellipse}\pysigline{\bfcode{ellipse}}
Whether or not the object should be an ellipse.

\end{fulllineitems}

\index{polygon (tmx.Object attribute)}

\begin{fulllineitems}
\phantomsection\label{index:tmx.Object.polygon}\pysigline{\bfcode{polygon}}
A list of coordinate pair tuples relative to the object's
position indicating the points of the object's representation as
a polygon.  Set to \code{None} to not represent the object as a
polygon.

\end{fulllineitems}

\index{polyline (tmx.Object attribute)}

\begin{fulllineitems}
\phantomsection\label{index:tmx.Object.polyline}\pysigline{\bfcode{polyline}}
A list of coordinate pair tuples relative to the object's
position indicating the points of the object's representation as
a polyline.  Set to \code{None} to not represent the object as
a polyline.

\end{fulllineitems}


\end{fulllineitems}

\index{ObjectGroup (class in tmx)}

\begin{fulllineitems}
\phantomsection\label{index:tmx.ObjectGroup}\pysiglinewithargsret{\strong{class }\code{tmx.}\bfcode{ObjectGroup}}{\emph{name}, \emph{color=None}, \emph{opacity=1}, \emph{visible=True}, \emph{offsetx=0}, \emph{offsety=0}, \emph{draworder=None}, \emph{properties=None}, \emph{objects=None}}{}~\index{name (tmx.ObjectGroup attribute)}

\begin{fulllineitems}
\phantomsection\label{index:tmx.ObjectGroup.name}\pysigline{\bfcode{name}}
The name of the object group.

\end{fulllineitems}

\index{color (tmx.ObjectGroup attribute)}

\begin{fulllineitems}
\phantomsection\label{index:tmx.ObjectGroup.color}\pysigline{\bfcode{color}}
A {\hyperref[index:tmx.Color]{\emph{\code{Color}}}} object indicating the color used to display the
objects in this group.  Set to \code{None} for no color
definition.

\end{fulllineitems}

\index{opacity (tmx.ObjectGroup attribute)}

\begin{fulllineitems}
\phantomsection\label{index:tmx.ObjectGroup.opacity}\pysigline{\bfcode{opacity}}
The opacity of the object group as a value from 0 to 1.

\end{fulllineitems}

\index{visible (tmx.ObjectGroup attribute)}

\begin{fulllineitems}
\phantomsection\label{index:tmx.ObjectGroup.visible}\pysigline{\bfcode{visible}}
Whether or not the object group is visible.

\end{fulllineitems}

\index{offsetx (tmx.ObjectGroup attribute)}

\begin{fulllineitems}
\phantomsection\label{index:tmx.ObjectGroup.offsetx}\pysigline{\bfcode{offsetx}}
Rendering offset for this layer in pixels.

\end{fulllineitems}

\index{offsety (tmx.ObjectGroup attribute)}

\begin{fulllineitems}
\phantomsection\label{index:tmx.ObjectGroup.offsety}\pysigline{\bfcode{offsety}}
Rendering offset for this layer in pixels.

\end{fulllineitems}

\index{draworder (tmx.ObjectGroup attribute)}

\begin{fulllineitems}
\phantomsection\label{index:tmx.ObjectGroup.draworder}\pysigline{\bfcode{draworder}}
Can be ``topdown'' or ``index''.  Set to \code{None} to not define
this.

\end{fulllineitems}

\index{properties (tmx.ObjectGroup attribute)}

\begin{fulllineitems}
\phantomsection\label{index:tmx.ObjectGroup.properties}\pysigline{\bfcode{properties}}
A list of {\hyperref[index:tmx.Property]{\emph{\code{Property}}}} objects indicating the object group's
properties

\end{fulllineitems}



\begin{fulllineitems}
\pysigline{\bfcode{objects:}}
A list of {\hyperref[index:tmx.Object]{\emph{\code{Object}}}} objects indicating the object group's
objects.

\end{fulllineitems}


\end{fulllineitems}

\index{Property (class in tmx)}

\begin{fulllineitems}
\phantomsection\label{index:tmx.Property}\pysiglinewithargsret{\strong{class }\code{tmx.}\bfcode{Property}}{\emph{name}, \emph{value}}{}~\index{name (tmx.Property attribute)}

\begin{fulllineitems}
\phantomsection\label{index:tmx.Property.name}\pysigline{\bfcode{name}}
The name of the property.

\end{fulllineitems}

\index{value (tmx.Property attribute)}

\begin{fulllineitems}
\phantomsection\label{index:tmx.Property.value}\pysigline{\bfcode{value}}
The value of the property.

The following types are specially recognized by the TMX format
and preserved when saving:
\begin{itemize}
\item {} 
Integers

\item {} 
Floats

\item {} 
Booleans

\item {} 
{\hyperref[index:tmx.Color]{\emph{\code{Color}}}} objects

\item {} 
\code{pathlib.PurePath} objects

\end{itemize}

Any other type is implicitly converted to and stored as a string
when the TMX file is saved.

\end{fulllineitems}


\end{fulllineitems}

\index{TerrainType (class in tmx)}

\begin{fulllineitems}
\phantomsection\label{index:tmx.TerrainType}\pysiglinewithargsret{\strong{class }\code{tmx.}\bfcode{TerrainType}}{\emph{name}, \emph{tile}, \emph{properties=None}}{}~\index{name (tmx.TerrainType attribute)}

\begin{fulllineitems}
\phantomsection\label{index:tmx.TerrainType.name}\pysigline{\bfcode{name}}
The name of the terrain type.

\end{fulllineitems}

\index{tile (tmx.TerrainType attribute)}

\begin{fulllineitems}
\phantomsection\label{index:tmx.TerrainType.tile}\pysigline{\bfcode{tile}}
The local tile ID of the tile that represents the terrain
visually.

\end{fulllineitems}

\index{properties (tmx.TerrainType attribute)}

\begin{fulllineitems}
\phantomsection\label{index:tmx.TerrainType.properties}\pysigline{\bfcode{properties}}
A list of {\hyperref[index:tmx.Property]{\emph{\code{Property}}}} objects indicating the terrain type's
properties.

\end{fulllineitems}


\end{fulllineitems}

\index{Tile (class in tmx)}

\begin{fulllineitems}
\phantomsection\label{index:tmx.Tile}\pysiglinewithargsret{\strong{class }\code{tmx.}\bfcode{Tile}}{\emph{id\_}, \emph{terrain=None}, \emph{probability=None}, \emph{properties=None}, \emph{image=None}, \emph{animation=None}}{}~\index{id (tmx.Tile attribute)}

\begin{fulllineitems}
\phantomsection\label{index:tmx.Tile.id}\pysigline{\bfcode{id}}
The local tile ID within its tileset.

\end{fulllineitems}

\index{terrain (tmx.Tile attribute)}

\begin{fulllineitems}
\phantomsection\label{index:tmx.Tile.terrain}\pysigline{\bfcode{terrain}}
Defines the terrain type of each corner of the tile, given as
comma-separated indexes in the list of terrain types in the order
top-left, top-right, bottom-left, bottom-right.  Leaving out a
value means that corner has no terrain. Set to \code{None} for
no terrain.

\end{fulllineitems}

\index{probability (tmx.Tile attribute)}

\begin{fulllineitems}
\phantomsection\label{index:tmx.Tile.probability}\pysigline{\bfcode{probability}}
A percentage indicating the probability that this tile is chosen
when it competes with others while editing with the terrain tool.
Set to \code{None} to not define this.

\end{fulllineitems}

\index{properties (tmx.Tile attribute)}

\begin{fulllineitems}
\phantomsection\label{index:tmx.Tile.properties}\pysigline{\bfcode{properties}}
A list of {\hyperref[index:tmx.Property]{\emph{\code{Property}}}} objects indicating the tile's
properties.

\end{fulllineitems}

\index{image (tmx.Tile attribute)}

\begin{fulllineitems}
\phantomsection\label{index:tmx.Tile.image}\pysigline{\bfcode{image}}
An {\hyperref[index:tmx.Image]{\emph{\code{Image}}}} object indicating the tile's image.  Set to
\code{None} for no image.

\end{fulllineitems}

\index{animation (tmx.Tile attribute)}

\begin{fulllineitems}
\phantomsection\label{index:tmx.Tile.animation}\pysigline{\bfcode{animation}}
A list of \code{Frame} objects indicating this tile's animation.
Set to \code{None} for no animation.

\end{fulllineitems}


\end{fulllineitems}

\index{Tileset (class in tmx)}

\begin{fulllineitems}
\phantomsection\label{index:tmx.Tileset}\pysiglinewithargsret{\strong{class }\code{tmx.}\bfcode{Tileset}}{\emph{firstgid}, \emph{name}, \emph{tilewidth}, \emph{tileheight}, \emph{source=None}, \emph{spacing=0}, \emph{margin=0}, \emph{xoffset=0}, \emph{yoffset=0}, \emph{tilecount=None}, \emph{columns=None}, \emph{properties=None}, \emph{image=None}, \emph{terraintypes=None}, \emph{tiles=None}}{}~\index{firstgid (tmx.Tileset attribute)}

\begin{fulllineitems}
\phantomsection\label{index:tmx.Tileset.firstgid}\pysigline{\bfcode{firstgid}}
The first global tile ID of this tileset (this global ID maps to
the first tile in this tileset).

\end{fulllineitems}

\index{name (tmx.Tileset attribute)}

\begin{fulllineitems}
\phantomsection\label{index:tmx.Tileset.name}\pysigline{\bfcode{name}}
The name of this tileset.

\end{fulllineitems}

\index{tilewidth (tmx.Tileset attribute)}

\begin{fulllineitems}
\phantomsection\label{index:tmx.Tileset.tilewidth}\pysigline{\bfcode{tilewidth}}
The (maximum) width of the tiles in this tileset.

\end{fulllineitems}

\index{tileheight (tmx.Tileset attribute)}

\begin{fulllineitems}
\phantomsection\label{index:tmx.Tileset.tileheight}\pysigline{\bfcode{tileheight}}
The (maximum) height of the tiles in this tileset.

\end{fulllineitems}

\index{source (tmx.Tileset attribute)}

\begin{fulllineitems}
\phantomsection\label{index:tmx.Tileset.source}\pysigline{\bfcode{source}}
The external TSX (Tile Set XML) file to store this tileset in.
If set to \code{None}, this tileset is stored in the TMX file.

\end{fulllineitems}

\index{spacing (tmx.Tileset attribute)}

\begin{fulllineitems}
\phantomsection\label{index:tmx.Tileset.spacing}\pysigline{\bfcode{spacing}}
The spacing in pixels between the tiles in this tileset (applies
to the tileset image).

\end{fulllineitems}

\index{margin (tmx.Tileset attribute)}

\begin{fulllineitems}
\phantomsection\label{index:tmx.Tileset.margin}\pysigline{\bfcode{margin}}
The margin around the tiles in this tileset (applies to the
tileset image).

\end{fulllineitems}

\index{xoffset (tmx.Tileset attribute)}

\begin{fulllineitems}
\phantomsection\label{index:tmx.Tileset.xoffset}\pysigline{\bfcode{xoffset}}
The horizontal offset of the tileset in pixels (positive is
right).

\end{fulllineitems}

\index{yoffset (tmx.Tileset attribute)}

\begin{fulllineitems}
\phantomsection\label{index:tmx.Tileset.yoffset}\pysigline{\bfcode{yoffset}}
The vertical offset of the tileset in pixels (positive is down).

\end{fulllineitems}

\index{tilecount (tmx.Tileset attribute)}

\begin{fulllineitems}
\phantomsection\label{index:tmx.Tileset.tilecount}\pysigline{\bfcode{tilecount}}
The number of tiles in this tileset.  Set to \code{None} to not
specify this.

\end{fulllineitems}

\index{columns (tmx.Tileset attribute)}

\begin{fulllineitems}
\phantomsection\label{index:tmx.Tileset.columns}\pysigline{\bfcode{columns}}
The number of tile columns in the tileset.  Set to \code{None}
to not specify this.

\end{fulllineitems}

\index{properties (tmx.Tileset attribute)}

\begin{fulllineitems}
\phantomsection\label{index:tmx.Tileset.properties}\pysigline{\bfcode{properties}}
A list of {\hyperref[index:tmx.Property]{\emph{\code{Property}}}} objects indicating the tileset's
properties.

\end{fulllineitems}

\index{image (tmx.Tileset attribute)}

\begin{fulllineitems}
\phantomsection\label{index:tmx.Tileset.image}\pysigline{\bfcode{image}}
An {\hyperref[index:tmx.Image]{\emph{\code{Image}}}} object indicating the tileset's image.  Set to
\code{None} for no image.

\end{fulllineitems}

\index{terraintypes (tmx.Tileset attribute)}

\begin{fulllineitems}
\phantomsection\label{index:tmx.Tileset.terraintypes}\pysigline{\bfcode{terraintypes}}
A list of {\hyperref[index:tmx.TerrainType]{\emph{\code{TerrainType}}}} objects indicating the tileset's
terrain types.

\end{fulllineitems}

\index{tiles (tmx.Tileset attribute)}

\begin{fulllineitems}
\phantomsection\label{index:tmx.Tileset.tiles}\pysigline{\bfcode{tiles}}
A list of {\hyperref[index:tmx.Tile]{\emph{\code{Tile}}}} objects indicating the tileset's tile
properties.

\end{fulllineitems}


\end{fulllineitems}



\chapter{Functions}
\label{index:functions}\index{data\_decode() (in module tmx)}

\begin{fulllineitems}
\phantomsection\label{index:tmx.data_decode}\pysiglinewithargsret{\code{tmx.}\bfcode{data\_decode}}{\emph{data}, \emph{encoding}, \emph{compression=None}}{}
Decode encoded data and return a list of integers it represents.

This is a low-level function used internally by this library; you
don't typically need to use it.

Arguments:
\begin{itemize}
\item {} 
\code{data} -- The data to decode.

\item {} 
\code{encoding} -- The encoding of the data.  Can be \code{"base64"} or
\code{"csv"}.

\item {} 
\code{compression} -- The compression method used.  Valid compression
methods are \code{"gzip"} and \code{"zlib"}.  Set to \code{None} for
no compression.

\end{itemize}

\end{fulllineitems}

\index{data\_encode() (in module tmx)}

\begin{fulllineitems}
\phantomsection\label{index:tmx.data_encode}\pysiglinewithargsret{\code{tmx.}\bfcode{data\_encode}}{\emph{data}, \emph{encoding}, \emph{compression=True}}{}
Encode a list of integers and return the encoded data.

This is a low-level function used internally by this library; you
don't typically need to use it.

Arguments:
\begin{itemize}
\item {} 
\code{data} -- The list of integers to encode.

\item {} 
\code{encoding} -- The encoding of the data.  Can be \code{"base64"} or
\code{"csv"}.

\item {} 
\code{compression} -- Whether or not compression should be used if
supported.

\end{itemize}

\end{fulllineitems}



\chapter{Indices and tables}
\label{index:indices-and-tables}\begin{itemize}
\item {} 
\DUspan{xref,std,std-ref}{genindex}

\item {} 
\DUspan{xref,std,std-ref}{modindex}

\item {} 
\DUspan{xref,std,std-ref}{search}

\end{itemize}


\renewcommand{\indexname}{Python Module Index}
\begin{theindex}
\def\bigletter#1{{\Large\sffamily#1}\nopagebreak\vspace{1mm}}
\bigletter{t}
\item {\texttt{tmx}}, \pageref{index:module-tmx}
\end{theindex}

\renewcommand{\indexname}{Index}
\printindex
\end{document}
